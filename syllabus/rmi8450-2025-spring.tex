\documentclass[a4paper, 12pt]{article}

% Packages for layout and formatting
\usepackage[a4paper, total={6in, 8in}]{geometry}
\usepackage{xcolor}
\usepackage{textcomp}

% Packages for content
\usepackage{blindtext}
\usepackage{hyperref}
\usepackage{footnote}
\usepackage{tablefootnote}
\makesavenoteenv{tabular}

% Hyperref options
\hypersetup{
    colorlinks=true,
    linkcolor=blue,
    filecolor=magenta,      
    urlcolor=cyan,
    pdftitle={RMI 8450: Machine Learning Applications in Actuarial Science and Risk Management},
    pdfauthor={Xiangshi Yin},
    pdfpagemode=FullScreen,
}

\title{
    RMI 8450: Machine Learning Applications in Actuarial Science and Risk Management\\
    \large Syllabus for Spring 2025\footnote{The course syllabus provides a general plan for the course; deviations may be necessary.}
}
\author{Instructor: Xiangshi Yin\thanks{Email: \href{mailto:xyin@gsu.edu}{xyin@gsu.edu}}}

\begin{document}

\maketitle

\tableofcontents

\section{Description}
This course explores the application of machine learning algorithms in actuarial science and risk management. Students will learn theoretical foundations, model selection, evaluation techniques, and practical applications through case studies, preparing them to address real-world challenges in these fields.

\subsection{Instructor}
\begin{center}
  \begin{tabular}{ l | c r }
    \hline			
    Name & Xiangshi Yin\\
    Email & \href{mailto:xyin@gsu.edu}{xyin@gsu.edu}\\
    \hline  
  \end{tabular}
\end{center}

\subsection{Teaching Assistant}
\begin{center}
	\begin{tabular}{ l | c r }
		\hline			
		Name & TBD\\
		Email & N/A\\
		\hline  
	\end{tabular}
\end{center}

\subsection{Lectures}
We meet on every Monday evening at 7:15 PM (Eastern Time).
\begin{center}
  \begin{tabular}{ l | c r }
    \hline			
    Days & Monday\\
   Time & 7:15 PM - 9:45 PM (Eastern Time) \\
    Room & (Online) Webex@iCollege\\
    \hline  
  \end{tabular}
\end{center}
\begin{flushleft}
To attend the online class, you need to:
\begin{itemize}
  \item Go to the class home page on iCollege at \url{https://gastate.view.usg.edu/d2l/home/xxxxxxxtbd}
  \item Click \colorbox{lightgray}{Webex} tab\textrightarrow Click \colorbox{lightgray}{Virtual Meetings}\textrightarrow Choose the corresponding class link and join.
  \item You could also click \colorbox{lightgray}{Content} \colorbox{lightgray}{Course Schedule} \textrightarrow Choose the corresponding class link and join
\end{itemize}
\end{flushleft}

\subsection{Office Hours}
\begin{center}
  \begin{tabular}{ l | c r }
    \hline			
    Days & Monday\\
    Type & By Appointment\\
    Time Slot 1 & 6:30 PM - 6:45 PM (Eastern Time) \\
    Time Slot 2 & 6:50 PM - 7:05 PM (Eastern Time) \\
    Room & (Online) Webex@iCollege\\
    \hline  
  \end{tabular}
\end{center}

\begin{flushleft}
Please note that the office hours listed below are tentative and may be adjusted based on feasibility and student feedback. There are two 15-minute sessions available every Monday before our regular classes. To book a time slot, you need to:
  \begin{itemize}
    \item Go to the class home page on iCollege at \url{https://gastate.view.usg.edu/d2l/home/xxxxxxxtbd}
    \item Click \colorbox{lightgray}{Webex} tab\textrightarrow Click \colorbox{lightgray}{Office Hours}\textrightarrow Choose the available time slot and click \colorbox{lightgray}{Book}.
    \item After the meeting is booked, you'll receive Webex online meeting instructions in your school email address, and you can also add the meeting to your calendar so that you don't miss it.
  \end{itemize}
\end{flushleft}

\subsection{Contact the instructor}
During the term, it is highly recommended that you contact the instructor either during scheduled office hours or via email. The instructor is available to help you gain access to resources, focus your projects, and answer any questions you may have. Additionally, your classmates can also be a valuable source of assistance.

\subsection{Course Website}
All class information will be posted on the \href{https://icollege.gsu.edu/}{iCollege} site. This includes lecture notes, assignments, key announcements, and links to additional websites with course-related materials. Additionally, source code, data, and other resources used in the class can also be found on our \href{https://github.com/xiangshiyin/machine-learning-for-actuarial-science}{GitHub repository}.

\section{Overview}
In this course, we will explore the application of machine learning algorithms in actuarial science and risk management through common use cases such as:
\begin{itemize}
    \item Time series modeling
    \item Marketing campaign predictions
    \item Insurance claim predictions
    \item Credit risk modeling
    \item Operational risk modeling and fraud detection
    \item Natural Language Processing (NLP) and information extraction
\end{itemize}
We will begin each topic with an overview of the theoretical foundations of relevant statistical and machine learning models. Discussions will cover the pros and cons of each model, best practices for model selection and evaluation, and case studies demonstrating their real-world applications. For some classical models, we will also implement key algorithms from scratch to deepen our understanding. This approach aims to provide students with a comprehensive grasp of both the theoretical and practical aspects of these models.

\subsection{Intended Audience}
This course is designed for students who have a basic understanding of statistics and machine learning and are interested in applying these techniques to actuarial science and risk management. Basic knowledge of Python programming is required, and students should be comfortable with data manipulation and visualization using Python programming. We will do a quick survey of Python programming to ensure that everyone is on the same page but will not cover the basics in detail.

\subsection{Learning Objectives}

\end{document}