\documentclass[a4paper, 12pt]{article}

% Packages for layout and formatting
\usepackage[a4paper, total={6in, 8in}]{geometry}
\usepackage{xcolor}
\usepackage{textcomp}

% Packages for content
\usepackage{blindtext}
\usepackage{hyperref}
\usepackage{footnote}
\usepackage{tablefootnote}
\makesavenoteenv{tabular}

% Hyperref options
\hypersetup{
    colorlinks=true,
    linkcolor=blue,
    filecolor=magenta,      
    urlcolor=cyan,
    pdftitle={RMI 8450: Machine Learning Applications in Actuarial Science and Risk Management},
    pdfauthor={Xiangshi Yin},
    pdfpagemode=FullScreen,
}

\title{
    RMI 8450: Machine Learning Applications in Actuarial Science and Risk Management\\
    \large Syllabus for Spring 2025\footnote{The course syllabus provides a general plan for the course; deviations may be necessary.}
}
\author{Instructor: Xiangshi Yin\thanks{Email: \href{mailto:xyin@gsu.edu}{xyin@gsu.edu}}}

\begin{document}

\maketitle

\tableofcontents

\section{Description}
This course explores the application of machine learning algorithms in actuarial science and risk management. Students will learn theoretical foundations, model selection, evaluation techniques, and practical applications through case studies, preparing them to address real-world challenges in these fields.

\subsection{Instructor}
\begin{center}
  \begin{tabular}{ l | c r }
    \hline			
    Name & Xiangshi Yin\\
    Email & \href{mailto:xyin@gsu.edu}{xyin@gsu.edu}\\
    \hline  
  \end{tabular}
\end{center}

\subsection{Teaching Assistant}
\begin{center}
	\begin{tabular}{ l | c r }
		\hline			
		Name & TBD\\
		Email & N/A\\
		\hline  
	\end{tabular}
\end{center}

\subsection{Lectures}
We meet on every Monday evening at 7:15 PM (Eastern Time).
\begin{center}
  \begin{tabular}{ l | c r }
    \hline			
    Days & Monday\\
   Time & 7:15 PM - 9:45 PM (Eastern Time) \\
    Room & (Online) Webex@iCollege\\
    \hline  
  \end{tabular}
\end{center}
\begin{flushleft}
To attend the online class, you need to:
\begin{itemize}
  \item Go to the class home page on iCollege at \url{https://gastate.view.usg.edu/d2l/home/xxxxxxxtbd}
  \item Click \colorbox{lightgray}{Webex} tab\textrightarrow Click \colorbox{lightgray}{Virtual Meetings}\textrightarrow Choose the corresponding class link and join.
  \item You could also click \colorbox{lightgray}{Content} \colorbox{lightgray}{Course Schedule} \textrightarrow Choose the corresponding class link and join
\end{itemize}
\end{flushleft}

\subsection{Office Hours}
\begin{center}
  \begin{tabular}{ l | c r }
    \hline			
    Days & Monday\\
    Type & By Appointment\\
    Time Slot 1 & 6:30 PM - 6:45 PM (Eastern Time) \\
    Time Slot 2 & 6:50 PM - 7:05 PM (Eastern Time) \\
    Room & (Online) Webex@iCollege\\
    \hline  
  \end{tabular}
\end{center}

\begin{flushleft}
Please note that the office hours listed below are tentative and may be adjusted based on feasibility and student feedback. There are two 15-minute sessions available every Monday before our regular classes. To book a time slot, you need to:
  \begin{itemize}
    \item Go to the class home page on iCollege at \url{https://gastate.view.usg.edu/d2l/home/xxxxxxxtbd}
    \item Click \colorbox{lightgray}{Webex} tab\textrightarrow Click \colorbox{lightgray}{Office Hours}\textrightarrow Choose the available time slot and click \colorbox{lightgray}{Book}.
    \item After the meeting is booked, you'll receive Webex online meeting instructions in your school email address, and you can also add the meeting to your calendar so that you don't miss it.
  \end{itemize}
\end{flushleft}

\subsection{Contact the instructor}
During the term, it is highly recommended that you contact the instructor either during scheduled office hours or via email. The instructor is available to help you gain access to resources, focus your projects, and answer any questions you may have. Additionally, your classmates can also be a valuable source of assistance.

\subsection{Course Website}
All class information will be posted on the \href{https://icollege.gsu.edu/}{iCollege} site. This includes lecture notes, assignments, key announcements, and links to additional websites with course-related materials. Additionally, source code, data, and other resources used in the class can also be found on our \href{https://github.com/xiangshiyin/machine-learning-for-actuarial-science}{GitHub repository}.

\section{Overview}
In this course, we will explore the application of machine learning algorithms in actuarial science and risk management through common use cases such as:
\begin{itemize}
    \item Time series modeling
    \item Marketing campaign predictions
    \item Insurance claim predictions
    \item Credit risk modeling
    \item Operational risk modeling and fraud detection
    \item Natural Language Processing (NLP) and information extraction
\end{itemize}
We will begin each topic with an overview of the theoretical foundations of relevant statistical and machine learning models. Discussions will cover the pros and cons of each model, best practices for model selection and evaluation, and case studies demonstrating their real-world applications. For some classical models, we will also implement key algorithms from scratch to deepen our understanding. This approach aims to provide students with a comprehensive grasp of both the theoretical and practical aspects of these models.

\subsection{Intended Audience}
This course is designed for students who have a basic understanding of statistics and machine learning and are interested in applying these techniques to actuarial science and risk management. Basic knowledge of Python programming is required, and students should be comfortable with data manipulation and visualization using Python programming. We will do a quick survey of Python programming to ensure that everyone is on the same page but will not cover the basics in detail.

\subsection{Learning Objectives}
Upon successfully completing this course, students will gain the following knowledge and skills:
\begin{itemize}
    \item Understand the theoretical foundations of machine learning algorithms that could be applied in actuarial science and risk management
    \item Develop the ability to select and evaluate machine learning models for different use cases
    \item Apply machine learning algorithms to some real-world problems 
\end{itemize}


\section{Course Schedule}
The course schedule is shown below. However, the topics are subject to change based on the pace of the class and the feedbacks of the students. Please refer to the iCollege site for the most up-to-date information.

\begin{center}
    \begin{tabular}{ l | c | c }
        \hline			
        Week & Date & Topic\\
        \hline
        01 & 2025-01-13 & Introduction to Python for Machine Learning\\
        02 & 2025-01-20 & (No Class)\footnote{Martin Luther King Jr. Day}\\
        03 & 2025-01-27 & Machine Learning Basics\\
        04 & 2025-02-03 & Introduction to Risk Management\\
        05 & 2025-02-10 & Data Exploration and Feature Engineering\\
        06 & 2025-02-17 & Evaluating Model Performance\\
        07 & 2025-02-24 & Time Series Modeling\\
        08 & 2025-03-03 & Credit Risk Modeling\\
        09 & 2025-03-10 & Insurance Claim Predictions\\
        10 & 2025-03-17 & (No Class)\footnote{Spring Break}\\
        11 & 2025-03-24 & Fraud Detections\\
        12 & 2025-03-31 & Marketing Campaign Predictions\\
        13 & 2025-04-07 & NLP and Information Extraction (part I)\\
        14 & 2025-04-14 & NLP and Information Extraction (part II)\\
        15 & 2025-04-21 & GPT Models and Prompt Engineering\\
        16 & 2025-04-28 & Final Project Presentation\\
        \hline  
    \end{tabular}
\end{center}

\section{Readings}
\subsection{Primary References}
\begin{itemize}
    \item \textbf{Hands-On Machine Learning with Scikit-Learn and Tensorflow} by Aur\'elien G\'eron
    \begin{itemize}
        \item Released October 2022
        \item Publisher(s): O'Reilly Media, Inc.
        \item ISBN: 9781098125974
    \end{itemize}

    \item \textbf{An Introduction to Statistical Learning, with Application in Python}
    \begin{itemize}
        \item Published July 2023 [\href{www.statlearning.com}{link}]
        \item Publisher(s): Springer
        \item ISBN: 9783031387463
        \item \href{https://www.statlearning.com/resources-python}{Lab Resources}
    \end{itemize}

    \item \textbf{Machine Learning for Financial Risk Management with Python}
    \begin{itemize}
        \item Published December 2021
        \item Publisher(s): O'Reilly Media, Inc.
        \item ISBN: 9781492085256
    \end{itemize}

    \item \textbf{Deep Credit Risk: Machine Learning with Python}
    \begin{itemize}
        \item Published June 2020
        \item Publisher(s): Independently published
        \item ISBN: 9798617590199
    \end{itemize}
\end{itemize}

\subsection{Other books and resources}
  \begin{itemize}
  	\item Fabrizio Romano  \textit{Learning Python: Learn to code like a professional with Python - an open source, versatile, and powerful programming language} Packt Publishing, 2015.
  	\item Charles Severance \textit{Python for Everybody: Exploring Data in Python 3} CreateSpace Independent Publishing Platform, 2016 
    \item Joel Grus \textit{Data Science from Scratch: First Principles with Python} O'Reilly Media, 2015.
    \item Foster Provost \textit{Data Science for Business: What You Need to Know about Data Mining and Data-Analytic Thinking} O'Reilly Media, 2013.
    \item \textit{An Introduction to Statistical Learning, with Application in Python} \url{https://hastie.su.domains/}
    \item Technical blogs posted on \url{www.medium.com}.
    \item Last but not least, \href{www.google.com}{Google} and \href{www.stackoverflow.com}{StackOverflow} are always your BEST FRIENDS to learn special coding skills.
  \end{itemize}

\section{Software}
\begin{itemize}
    \item All programming activities will be performed on the your own laptop. Your laptop should have Python 3 and Jupyter Notebook installed. Using the Anaconda installation (\url{https://docs.anaconda.com/anaconda/install}) is a good start to have most of the packages we need for the class in one shot. Detailed installation instructions will be posted on the class home page on \href{https://gastate.view.usg.edu/d2l/home/xxxxxxxtbd}{iCollege} and \href{https://github.com/xiangshiyin/machine-learning-for-actuarial-science}{our course GitHub repository}
    \item If you are interested to explore new tools, you could also try Google Colab (\url{colab.research.google.com}). It is an online Jupyter Notebook environment with Python and free computing resources backed by Google. You may need to install certain packages yourselves if they are not available in the notebook environment.
\end{itemize}  

\section{Homework/Quizzes/Final Project}
Homework are assigned once every 2 weeks. There will be 2 online quizzes and 1 final group project over the whole semester.

\subsection{Homework}
Homework assignments are the continuation of a hands-on activities in class. Detailed information about the activity and expectation for successful completion are provided with the instructions. See the web site for the most recent and detailed information on these assignments. \textbf{Homeworks are individual assignments!} You may discuss the assignment with your classmates, but your final answers should reflect your individual effort. Completed assignments must be uploaded\footnote{Instructions on homework submission will be posted on the class home page on iCollege} by the deadline.

\subsection{Quizzes}
Quizzes will be given out after the regular class and comprise only a few questions. However, some questions may need some thinking and calculations.

\subsection{Final Project}
TBD

\end{document}