\documentclass[a4paper, 12pt]{article}

% Packages for layout and formatting
\usepackage[a4paper, total={6in, 8in}]{geometry}
\usepackage{xcolor}
\usepackage{textcomp}

% Packages for content
\usepackage{blindtext}
\usepackage{hyperref}
\usepackage{footnote}
\usepackage{tablefootnote}
\makesavenoteenv{tabular}

% Hyperref options
\hypersetup{
    colorlinks=true,
    linkcolor=blue,
    filecolor=magenta,      
    urlcolor=cyan,
    pdftitle={RMI 8450: Machine Learning Applications in Actuarial Science and Risk Management},
    pdfauthor={Xiangshi Yin},
    pdfpagemode=FullScreen,
}

\title{
    RMI 8450: Machine Learning Applications in Actuarial Science and Risk Management\\
    \large Syllabus for Spring 2025\footnote{The course syllabus provides a general plan for the course; deviations may be necessary.}
}
\author{Instructor: Xiangshi Yin\thanks{Email: \href{mailto:xyin@gsu.edu}{xyin@gsu.edu}}}

\begin{document}

\maketitle

\tableofcontents

\section{Description}
This course explores the application of machine learning algorithms in actuarial science and risk management. Students will learn theoretical foundations, model selection, evaluation techniques, and practical applications through case studies, preparing them to address real-world challenges in these fields.

\subsection{Instructor}
\begin{center}
  \begin{tabular}{ l | c r }
    \hline			
    Name & Xiangshi Yin\\
    Email & \href{mailto:xyin@gsu.edu}{xyin@gsu.edu}\\
    \hline  
  \end{tabular}
\end{center}

\subsection{Teaching Assistant}
\begin{center}
	\begin{tabular}{ l | c r }
		\hline			
		Name & TBD\\
		Email & N/A\\
		\hline  
	\end{tabular}
\end{center}

\subsection{Lectures}
We meet on every Monday evening at 7:15 PM (Eastern Time).
\begin{center}
  \begin{tabular}{ l | c r }
    \hline			
    Days & Monday\\
   Time & 7:15 PM - 9:45 PM (Eastern Time) \\
    Room & (Online) Webex@iCollege\\
    \hline  
  \end{tabular}
\end{center}
\begin{flushleft}
To attend the online class, you need to:
\begin{itemize}
  \item Go to the class home page on iCollege at \url{https://gastate.view.usg.edu/d2l/home/xxxxxxxtbd}
  \item Click \colorbox{lightgray}{Webex} tab\textrightarrow Click \colorbox{lightgray}{Virtual Meetings}\textrightarrow Choose the corresponding class link and join.
  \item You could also click \colorbox{lightgray}{Content} \colorbox{lightgray}{Course Schedule} \textrightarrow Choose the corresponding class link and join
\end{itemize}
\end{flushleft}

\subsection{Office Hours}
\begin{center}
  \begin{tabular}{ l | c r }
    \hline			
    Days & Monday\\
    Type & By Appointment\\
    Time Slot 1 & 6:30 PM - 6:45 PM (Eastern Time) \\
    Time Slot 2 & 6:50 PM - 7:05 PM (Eastern Time) \\
    Room & (Online) Webex@iCollege\\
    \hline  
  \end{tabular}
\end{center}

\begin{flushleft}
Please note that the office hours listed below are tentative and may be adjusted based on feasibility and student feedback. There are two 15-minute sessions available every Monday before our regular classes. To book a time slot, you need to:
  \begin{itemize}
    \item Go to the class home page on iCollege at \url{https://gastate.view.usg.edu/d2l/home/xxxxxxxtbd}
    \item Click \colorbox{lightgray}{Webex} tab\textrightarrow Click \colorbox{lightgray}{Office Hours}\textrightarrow Choose the available time slot and click \colorbox{lightgray}{Book}.
    \item After the meeting is booked, you'll receive Webex online meeting instructions in your school email address, and you can also add the meeting to your calendar so that you don't miss it.
  \end{itemize}
\end{flushleft}

\subsection{Contact the instructor}
During the term, it is highly recommended that you contact the instructor either during scheduled office hours or via email. The instructor is available to help you gain access to resources, focus your projects, and answer any questions you may have. Additionally, your classmates can also be a valuable source of assistance.

\subsection{Course Website}
All class information will be posted on the \href{https://icollege.gsu.edu/}{iCollege} site. This includes lecture notes, assignments, key announcements, and links to additional websites with course-related materials. Additionally, source code, data, and other resources used in the class can also be found on our \href{https://github.com/xiangshiyin/machine-learning-for-actuarial-science}{GitHub repository}.

\section{Overview}
In this course, we will explore the application of machine learning algorithms in actuarial science and risk management through common use cases such as:
\begin{itemize}
    \item Time series modeling
    \item Marketing campaign predictions
    \item Insurance claim predictions
    \item Credit risk modeling
    \item Operational risk modeling and fraud detection
    \item Natural Language Processing (NLP) and information extraction
\end{itemize}
We will begin each topic with an overview of the theoretical foundations of relevant statistical and machine learning models. Discussions will cover the pros and cons of each model, best practices for model selection and evaluation, and case studies demonstrating their real-world applications. For some classical models, we will also implement key algorithms from scratch to deepen our understanding. This approach aims to provide students with a comprehensive grasp of both the theoretical and practical aspects of these models.

\subsection{Intended Audience}
This course is designed for students who have a basic understanding of statistics and machine learning and are interested in applying these techniques to actuarial science and risk management. Basic knowledge of Python programming is required, and students should be comfortable with data manipulation and visualization using Python programming. We will do a quick survey of Python programming to ensure that everyone is on the same page but will not cover the basics in detail.

\subsection{Learning Objectives}
Upon successfully completing this course, students will gain the following knowledge and skills:
\begin{itemize}
    \item Understand the theoretical foundations of machine learning algorithms that could be applied in actuarial science and risk management
    \item Develop the ability to select and evaluate machine learning models for different use cases
    \item Apply machine learning algorithms to some real-world problems 
\end{itemize}


\section{Course Schedule}
The course schedule is shown below. However, the topics are subject to change based on the pace of the class and the feedbacks of the students. Please refer to the iCollege site for the most up-to-date information.

\begin{center}
    \begin{tabular}{ l | c | c }
        \hline			
        Week & Date & Topic\\
        \hline
        01 & 2025-01-13 & Introduction to Python for Machine Learning\\
        02 & 2025-01-20 & (No Class)\footnote{Martin Luther King Jr. Day}\\
        03 & 2025-01-27 & Machine Learning Basics\\
        04 & 2025-02-03 & Introduction to Risk Management\\
        05 & 2025-02-10 & Data Exploration and Feature Engineering\\
        06 & 2025-02-17 & Evaluating Model Performance\\
        07 & 2025-02-24 & Time Series Modeling\\
        08 & 2025-03-03 & Credit Risk Modeling\\
        09 & 2025-03-10 & Insurance Claim Predictions\\
        10 & 2025-03-17 & (No Class)\footnote{Spring Break}\\
        11 & 2025-03-24 & Fraud Detections\\
        12 & 2025-03-31 & Marketing Campaign Predictions\\
        13 & 2025-04-07 & NLP and Information Extraction (part I)\\
        14 & 2025-04-14 & NLP and Information Extraction (part II)\\
        15 & 2025-04-21 & GPT Models and Prompt Engineering\\
        16 & 2025-04-28 & Final Project Presentation\\
        \hline  
    \end{tabular}
\end{center}

\section{Readings}
\subsection{Primary References}
\begin{itemize}
    \item \textbf{Hands-On Machine Learning with Scikit-Learn and Tensorflow} by Aur\'elien G\'eron
    \begin{itemize}
        \item Released October 2022
        \item Publisher(s): O'Reilly Media, Inc.
        \item ISBN: 9781098125974
    \end{itemize}

    \item \textbf{An Introduction to Statistical Learning, with Application in Python}
    \begin{itemize}
        \item Published July 2023 [\href{www.statlearning.com}{link}]
        \item Publisher(s): Springer
        \item ISBN: 9783031387463
        \item \href{https://www.statlearning.com/resources-python}{Lab Resources}
    \end{itemize}

    \item \textbf{Machine Learning for Financial Risk Management with Python}
    \begin{itemize}
        \item Published December 2021
        \item Publisher(s): O'Reilly Media, Inc.
        \item ISBN: 9781492085256
    \end{itemize}

    \item \textbf{Deep Credit Risk: Machine Learning with Python}
    \begin{itemize}
        \item Published June 2020
        \item Publisher(s): Independently published
        \item ISBN: 9798617590199
    \end{itemize}
\end{itemize}

\subsection{Other books and resources}
  \begin{itemize}
  	\item Fabrizio Romano  \textit{Learning Python: Learn to code like a professional with Python - an open source, versatile, and powerful programming language} Packt Publishing, 2015.
  	\item Charles Severance \textit{Python for Everybody: Exploring Data in Python 3} CreateSpace Independent Publishing Platform, 2016 
    \item Joel Grus \textit{Data Science from Scratch: First Principles with Python} O'Reilly Media, 2015.
    \item Foster Provost \textit{Data Science for Business: What You Need to Know about Data Mining and Data-Analytic Thinking} O'Reilly Media, 2013.
    \item \textit{An Introduction to Statistical Learning, with Application in Python} \url{https://hastie.su.domains/}
    \item Technical blogs posted on \url{www.medium.com}.
    \item Last but not least, \href{www.google.com}{Google} and \href{www.stackoverflow.com}{StackOverflow} are always your BEST FRIENDS to learn special coding skills.
  \end{itemize}

\section{Software}
\begin{itemize}
    \item All programming activities will be performed on the your own laptop. Your laptop should have Python 3 and Jupyter Notebook installed. Using the Anaconda installation (\url{https://docs.anaconda.com/anaconda/install}) is a good start to have most of the packages we need for the class in one shot. Detailed installation instructions will be posted on the class home page on \href{https://gastate.view.usg.edu/d2l/home/xxxxxxxtbd}{iCollege} and \href{https://github.com/xiangshiyin/machine-learning-for-actuarial-science}{our course GitHub repository}
    \item If you are interested to explore new tools, you could also try Google Colab (\url{colab.research.google.com}). It is an online Jupyter Notebook environment with Python and free computing resources backed by Google. You may need to install certain packages yourselves if they are not available in the notebook environment.
\end{itemize}  

\section{Homework/Quizzes/Final Project}
Homework are assigned once every 2 weeks. There will be 2 online quizzes and 1 final group project over the whole semester\footnote{Schedule here is subject to change based on the pace of the class}.
\begin{center}
  \begin{tabular}{ l|l|c|c|l }
      \hline			
      Date & Note & Quiz & Homework & Due (EST)\\
      \hline
      2025-01-13 & First class & Y &  & 2025-01-27 19:15 \\
      2025-01-20 & (No Class) &  &  & N/A \\
      2025-01-27 &  &  & Y & 2025-02-10 19:15 \\
      2025-02-03 &  & Y &  & 2025-02-10 19:15 \\
      2025-02-10 &  &  & Y & 2025-02-24 19:15 \\
      2025-02-17 &  &  &  & N/A \\
      2025-02-24 &   &  & Y & 2025-03-10 19:15 \\
      2025-03-03 & Team proposals &  &  & N/A \\
      2025-03-10 & Team finalized &  & Y & 2025-03-24 19:15 \\
      2025-03-17 & (No Class) &  &  & N/A \\
      2025-03-24 & Suggested project topics &  & Y & 2025-04-07 19:15 \\
      2025-03-31 &  &  &  & N/A \\
      2025-04-07 & Project topics finalized &  & Y & 2025-04-21 19:15 \\
      2025-04-14 &  &  &  & N/A \\
      2025-04-21 &  &  &  & N/A \\
      2025-04-28 & Final Project Presentation &  &  & N/A \\
      \hline  
  \end{tabular}
\end{center}

\subsection{Homework}
Homework assignments are the continuation of a hands-on activities in class. Detailed information about the activity and expectation for successful completion are provided with the instructions. See the web site for the most recent and detailed information on these assignments. \textbf{Homeworks are individual assignments!} You may discuss the assignment with your classmates, but your final answers should reflect your individual effort. Completed assignments must be uploaded\footnote{Instructions on homework submission will be posted on the class home page on iCollege} by the deadline.

\subsection{Quizzes}
Quizzes will be given out after the regular class and comprise only a few questions. However, some questions may need some thinking and calculations.

\subsection{Final Project}
The project has to showcase a subset of the methodologies and techniques covered in the course. Teams can comprise up to 3 students, and should form by the expected date. \textbf{Teams are free to choose a data set for their project} (the instructor will also release a list of suggested topics as a reference). The use of proprietary or classified data sets is not allowed. \textbf{Project deliverables include a detailed report, functioning code, and a presentations.} Details about requirements and evaluation criteria will be posted on the class homepage on iCollege.\\
\\
Key dates to remember (also marked in the schedule above):\\
\begin{itemize}
  \item \textbf{2025-03-03}: Team proposals due
  \item \textbf{2025-03-10}: Team finalized
  \item \textbf{2025-03-24}: Suggested project topics due
  \item \textbf{2025-04-07}: Project topics finalized
  \item \textbf{2025-04-28}: Final project presentation
\end{itemize}

\begin{flushleft}
	\textbf{Teams will submit one assignment for all team members. In most cases, each member of the team will get the same score. Each team assignment must also include a list of tasks completed by each member.}
\end{flushleft}

\section{Evaluation}
Students will be evaluted by the deliverables summarized below:\\
\begin{center}
	\begin{tabular}{lcr}
		\hline
		Assignment & Percentage \\
		\hline
		Quizzes & 10\%\\
		Homework & 45\%\\
		Final Project & 45\%\\
		\hline
		Total & 100\%\\   
		\hline     
	\end{tabular}
\end{center}

\begin{center}
	\begin{tabular}{lcr}
		\hline
		Grade & Percentage\\
		\hline
		A+ & $\geq$ 97\\
		A & $\geq$ 90\\
		A- & $\geq$ 87\\
		B+ & $\geq$ 83\\
		B & $\geq$ 80\\
		B- & $\geq$ 77\\
		C+ & $\geq$ 73\\
		C & $\geq$ 70\\
		C- & $\geq$ 67\\
		D & $\geq$ 60\\
		F & $<$ 60\\
		\hline 
	\end{tabular}
\end{center}


\section{Workload Expectations}
Students should plan for 2 - 3 hours of work outside of class each week for each course credit hour. Thus, a 3-credit course averages between 6 and 9 hours of student work outside of the classroom, each week.
\\
\\
\textbf{Arbitration}: There will be a one-week arbitration period after graded activities are returned. Within that one-week period, you are encouraged to discuss any assumptions and/or misinterpretations that you made on the activity that may have influenced your grade.
\\
\\
\textbf{Attendance}: If you are unable to attend a class session, it is your responsibility to acquire the class notes, assignments, announcements, etc. from a classmate. The instructor will not give private lectures for those that miss class.
\\
\\
\textbf{Submission of Deliverables}: Unless specific, prior approval is obtained, no deliverable will be accepted after the specified due date. If you have a legitimate personal emergency (e.g., health problem) that may impair your ability to submit a deliverable on time, you must take the initiative to contact the instructor before the due date/time (or as soon after your emergency as possible) to communicate the situation. 

\section{General GSU Policies}
\subsection{Students with Disabilities or Special Needs}
Students who wish to request an accommodation for a disability may do so by registering with the GSU Access and Accommodations Center (AACE).   Students may only be accommodated upon issuance by the AACE of a signed Accommodation Plan and are responsible for providing a copy of that plan to instructors of all classes in which accommodations are sought.  Please let me know if you have a disability or special need that requires accommodation.

\subsection{Religious Accommodation and Holidays}
Students must provide instructors with reasonable notice of the dates of religious holidays on which they plan to be absent and must be given an equivalent opportunity to make up missed work according to an agreed-upon schedule. Such accommodations might include rescheduling an exam or giving the student a make-up exam, allowing an individual or group presentation to be made on a different date, letting a student attend a different section for the same class that week, adjusting a due date or assigning the student appropriate make-up work that is no more difficult than the original assignment. Students wishing to have an excused absence due to the observation of a religious holiday of special importance must provide \textbf{“advance written request to each instructor by the end of the first week of classes.”}

\subsection{GSU Policy on Withdrawing from Classes}
The semester midpoint (March 04, 2025) is the last day to voluntarily withdraw from a full semester class and receive a possible grade of W.  Withdrawals appear on the student’s permanent record and count toward their attempted hours.  
\\
\\
Students can use PAWS to withdraw before the midpoint; after that date, voluntary withdrawals cannot occur.  Students are allowed only 6 withdrawals during their academic careers at GSU. If they withdraw from your course after drop-add and before the midpoint, they receive a W (unless they already have 6 withdrawals); if they withdraw after the midpoint, they will automatically receive a WF.   After 6 withdrawals, withdrawal at any point in the course results in an automatic F. 
\\
\\
While Voluntary Withdrawals are the most common, GSU policy also permits Involuntary Withdrawals, Emergency Withdrawals, Military Withdrawals, and Non-Academic Withdrawals, and explains when and how students can initiate a withdrawal.   You are responsible for understanding and adhering to the GSU Revision of Class Schedule (Add/Drop and Withdrawal) policy explained in \href{https://catalogs.gsu.edu/content.php?catoid=13&navoid=1563}{Section 1332} of the Undergraduate Catalog.  

\subsection{Campus Safety APP (\href{https://safety.gsu.edu/livesafe/}{Livesafe Mobile App})}
Georgia State University values the safety of all university community members.  To promote campus safety, the university is providing the LiveSafe app free for all students, faculty, and staff.  This app provides a quick, convenient, and discrete way to communicate with the GSU police.  I strongly recommend that you download the app from either the Apple App Store or Google Play.  You can sign-up for Panther Alerts and learn more about LiveSafe by visiting the GSU \href{https://safety.gsu.edu/livesafe/}{LiveSafe webpage}.

\subsection{Campus Police Numbers}
\textbf{Please make sure you have these campus police numbers in your phone}
\begin{itemize}
  \item For emergencies call 404-413-3333
  \item For non-emergencies and to request a safety escort call 404-413-2100 
  \item If you are hearing impaired call 404-413-3203
\end{itemize}

\subsection{Family Educational Rights Privacy Act (FERPA)}
In keeping with USG and university policy, this course website will make every effort to maintain the privacy and accuracy of your personal information. Specifically, unless otherwise noted, it will not actively share personal information gathered from the site with anyone except university employees whose responsibilities require access to said records. However, some information collected from the site may be subject to the Georgia Open Records Act. This means that while we do not actively share information, in some cases we may be compelled by law to release information gathered from the site. Also, the site will be managed in compliance with the Family Educational Rights and Privacy Act (FERPA), which prohibits the release of education records without student permission.  For more details on FERPA, go \href{https://registrar.gsu.edu/academic-records/records-access/#ferpa}{here}.

\subsection{Course Assessment}
"Your constructive assessment of this course plays an indispensable role in shaping education at Georgia State.  Upon completing the course, please take the time to fill out the online course evaluation."

\subsection{Academic Honesty}
The \href{https://provost.gsu.edu/document/academic-honesty-policy/}{GSU Policy on Academic Honesty} applies to all your GSU courses, including LGLS 3610.  The policy defines and provides examples of several types of academic dishonesty, including plagiarism, cheating on exams, unauthorized collaboration, falsification, and multiple submissions.  In addition, the policy outlines the possible consequences for academic dishonesty, including failing the plagiarized assignment, failing the course, an annotation on your transcript, and even expulsion from the 
university.

\subsection{Student Code of Conduct (2024-2025)}
The \href{https://codeofconduct.gsu.edu/}{Student Code of Conduct} addresses a number of issues, including general student conduct that is prohibited by the university, disruptive student conduct, the university's non-discrimination policy, and the sexual misconduct policy.  For more information on any of these policies refer to the Code of Conduct.


\end{document}